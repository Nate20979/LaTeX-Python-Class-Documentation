\documentclass[12pt, letterpaper]{report}
    \usepackage[utf8]{inputenc}
    \usepackage{graphicx}
    \usepackage{pst-plot}
    \usepackage{geometry}
    \usepackage{fancyhdr}
    \usepackage{imakeidx}
    \geometry{letterpaper, portrait, right=1.25in, left=1.25in, top=1.0in, bottom=1.0in}
    \title{\Huge{Unbearable Lighteness of Being Written Assignment}}
    \author{Nathan Stern}
    \date{March 2018}
    \linespread{2}
    \pagenumbering{arabic}
    \makeindex
\begin{document}
    \graphicspath{/}
    \maketitle

    Throughout the novel, “The Unbearable Lightness of Being”, Kundera brings up the idea of Fidelity and Betrayal quite frequently. In fact, the nature of some of the characters in the novel are driven by a strong belief in fidelity and distaste of betrayal. Using the idea of fidelity and betrayal, Kundera shows both conflict and the true nature of characters in the novel. The conflict that he shows is not only between different characters, but also between characters and themselves. Be it the conflict between Tomas and Tereza, or Sabina and Franz, Kundera shows the nature of his characters through their fidelity to each other, and their eventual betrayal.\\*
    
    Tomas is a character that has a hard time having both love and sex under the same roof. He has the idea that love and sex don’t necessarily have to go hand in hand. This is what causes the main conflict between Tomas and Tereza. “Making love with a woman and sleeping with a woman are two separate passions, not merely different but opposite.” (Kundera, 15). This quote helps to show the true nature of Tomas. He is a character that is considered, by nature, to be light. This means that he is a character that doesn’t get weighed down by things like love or attachment to a significant other. This is very clear by how he decides to leave his first wife and never see his son again. “...Tomas decided on the spur of the moment never to see him again.” (Kundera, 11). This abrupt decision is yet another example of the betrayal of somebody close to him. It is also another example showing his light nature. This light nature that Kundera portrays is what drives his character to betray the ones closest to him. \\*
    
    Tereza is a character that cannot fathom the idea of being unfaithful and is considered to be a heavy character in the novel. Her early life, and her mother’s influence on her early years, has caused her to be sensitive around the idea of her body and the people around her. This is one of the main causes of her anger directed towards Tomas when she finds out about his mistress, Sabina. “...in Tereza’s eyes, the stigma of his exploits with the mistresses.” This shows that Tereza hold the fact that Tomas has mistresses against him. Later on in the novel, after Tomas and Tereza get married and they move out of Prague, Tereza leaves a letter for Tomas on his desk and leaves with Karenin for Prague. The one time that Tereza experiments with being unfaithful, it goes horribly wrong. “She thrashed in his arms, swung her fists in the air, and spat in his face.” This shows that even though she had the intentions of exposing herself to Tomas’s world, she could not actually condone to the involuntary actions of her body due to the actions of the engineer she was trying to have an affair with. Tereza, as written by Kundera, is an inherently heavy character and the conflict between her and Tomas further solidifies that fact.\\*
    
    The conflict between Tomas and Tereza is a very good example of the way Kundera uses the idea of fidelity and betrayal to show the nature of his characters. Their relationship, in the novel, is shrouded in secrecy, and doubt. Tomas is extremely unfaithful while Tereza is basically the exact opposite. Tomas and Tereza balance each other out in terms of their natures. Tereza acts as an anchor for Tomas’s various affairs. “He was caught in a trap: even on his way to see them, he found them distasteful, but one day without them and he was back on the phone, eager to make contact.” (Kundera, 22). This shows that even though Tomas wanted to have his affairs, there was a force that Tereza seemed to generate that pulled him back to her. This is what drove them to change the way they were at the end of the novel. Tomas tried to become a heavier person, in nature, and started to plant his roots in one place. Tereza tried to become a lighter person and find other people that she could interact with.\\*
    
    Sabina is very much like Tomas. She is a free soul that does whatever she wants, whenever she wants to. Sabina’s nature, as written by Kundera, is light. This means that she pays no mind to the ideas of fidelity and betrayal. She just ends up doing whatever she wants to. “And again she felt a longing to betray: betray her own betrayal.” (Kundera, 92). This shows that not only does Sabina pay no mind to betraying those closest to her, she longs to betray the people closest to her. It also shows that no matter how rooted she seems to be in a given situation, she will find a way to remove herself. At one point in the novel, she says that she tries not to become too attached to anybody. As such, when Franz tells Sabina that he left his wife to be with her, she leaves him behind and moves onto a new life yet again. Throughout the novel, Sabina continually repeats the same actions, over and over again. She finds somebody that she wants to have an affair with, she becomes rooted in their lives, and when they become too attached to her she leaves. This shows that Sabina has reached the “Unbearable Lightness of Being” that is described in the book. The lightness that Kundera portrays in Sabina is very similar to that of Tomas. The main difference is the fact that Sabina uproots her life whenever she feels she needs to, and Tomas, to a certain extent, stays faithful to Tereza and stays in one place.\\*
    
    Franz is very similar to Tereza in many ways. He has trouble dealing with infidelity, specifically his own. Franz was practically tortured by the fact that he had to betray his wife in order to be with Sabina. When he was certain that he loved Sabina, he left his wife to be with her. “...he assumed that Sabina would be charmed by his ability to be faithful, that it would win her over.” (Kundera, 91). This shows that Franz has a deep rooted belief in the idea of Fidelity. This is the exact opposite to the woman, Sabina, that he had planned on spending the rest of his life with. His fidelity can then be seen again later in the book when he is seeing one of his students who was named, rather accurately by Kundera, Young Student. He also, later in the book, showed that he was still being loyal to Sabina by participating in a march to the Cambodian border and eventually being killed while trying to do what he thought Sabina would do. “If he went on the march, Sabina would gaze down on him enraptured; she would understand that he had remained faithful to her.” (Kundera, 259). This continues to prove that Franz is a character that Kundera chose to contrast the nature of Sabina. Franz is a character that would be considered as heavy due to the fact that he feels the need to stay loyal to the people closest to him. After Sabina leaves him, he decides to try to live a life that would be considered lighter. However, as previously mentioned, he ended up trying to prove that he was still faithful to Sabina and ultimately failed his mission to become lighter.\\*
    
    The conflict caused due to the different natures of Sabina and Franz is quite influential. After leaving Franz, Sabina decided to try not to betray those around her as frequently. This was basically a 180 degree turn away from who she was before the endeavor. The same can be said about Franz. He turned his life around and tried to be a lighter person as well as trying new things. The main point, though, is that they decided to try to change. They ultimately failed in their endeavors and went back to what they were like before.\\*
    
    Fidelity and Betrayal play a large role in Kundera’s development of his characters throughout the novel. His ideas not only help to reveal the nature of the characters that he chooses to show in the novel, but also helps to shed some light on the idea of human nature. The four main characters in the novel decide, at some point, to try to change their nature, be it light to heavy or heavy to light. Kundera makes a statement about human nature through the outcome of these endeavors by the characters. He shows that the nature of humans is to remain as we are, no matter how hard we try to change. He shows this by having his characters fail to change, and eventually go back to the ways that they were trying to change. Kundera’s use of fidelity and betrayal throughout the novel not only shows the nature of his characters, but also portrays his idea that human nature is immutable.\\*

    \printindex
\end{document}